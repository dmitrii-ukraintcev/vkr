\section*{ОБОЗНАЧЕНИЯ И СОКРАЩЕНИЯ}

БД -- база данных.

ИС -- информационная система.

ИТ -- информационные технологии. 

%КТС -- комплекс технических средств.

%ОМТС -- отдел материально-технического снабжения. 

ПО -- программное обеспечение.

РП -- рабочий проект.

СУБД -- система управления базами данных.

ТЗ -- техническое задание.

ТП -- технический проект.

UML (Unified Modelling Language) -- язык графического описания для объектного моделирования в области разработки программного обеспечения.

ООП - (Объектно-ориентированное программирование) -- методология программирования, основанная на представлении программы в виде совокупности объектов.

HTTP (HyperText Transfer Protocol) -- протокол прикладного уровня передачи данных.

HTML (HyperText Markup Language) -- язык гипертекстовой разметки документов для просмотра веб-страниц в браузере.

SQL (Structured Query Language) -- декларативный язык программирования для структурных. запросов к базе данных.

CMS (Content Management System) -- система управления содержимым.

MVC (Model-View-Controller) -- это шаблон (паттерн) программирования, разделяющий архитектуру приложения на три отдельных компонента: модель (Model), представление (View), контроллер (Controller).

ORM (Object-Relational Mapping) -- технология программирования, которая связывает базы данных с концепциями объектно-ориентированных языков программирования.

WYSIWYG (What You See Is What You Get, «что видишь, то и получишь») -- свойство прикладных программ или веб-интерфейсов, в которых содержание отображается в процессе редактирования и выглядит максимально близко похожим на конечную продукцию, которая может быть печатным документом, веб-страницей или презентацией.
