\section*{ВВЕДЕНИЕ}
\addcontentsline{toc}{section}{ВВЕДЕНИЕ}

Любой веб-сайт состоит из набора страниц, а различия заключаются лишь в том, как они организованы. Существует два вида организации веб-сайта -- статический и динамический. В первом случае специалисты, отвечающие за создание и поддержку сайта пишут в HTML-форме каждую в отдельности страницу, включая ее оформление и контент. Во втором -- в основе любой веб-страницы лежит шаблон, определяющий расположение в окне веб-браузера всех компонентов страницы, и вставка конкретной информации производится с использованием стандартных средств, не требующих от участника процесса знания языка HTML и достаточно сложных для неспециалиста процедур публикации веб-страницы.

Если сайт состоит из множества страниц или он должен часто обновляться, то преимущество динамической организации становится очевидным. Разработчикам веб-сайта не надо переписывать всю страницу при изменении ее информационного наполнения или дизайна. Страницы не хранятся целиком, а формируются динамически при обращении к ним.

Таким образом, отделение дизайна от контента является главной отличительной особенностью динамических сайтов от статических. На этой основе возможны дальнейшие усовершенствования структуры сайта, такие как определение различных пользовательских функций и автоматизация бизнес-процессов, а самое главное, контроль поступающего на сайт контента.

Для создания динамического сайта возможны два пути. Во-первых, это написание собственных программ, отвечающих за создание нужных шаблонов и поддерживающих необходимые функции. При этом созданная система будет полностью отвечать потребностям, однако возможно потребует больших программистских усилий и времени. Второй путь - это воспользоваться уже существующими системами, которые и называются системами управления веб-контентом. Преимуществом этого пути является уменьшение затрат времени и сил. К его недостаткам можно отнести снижение гибкости, предоставление недостаточного или чрезмерного набора возможностей.

Под контентом (содержимым) понимают информационное наполнение сайта -- то есть все типы материалов, которые находятся на сервере: веб-страницы, документы, программы, аудио-файлы, фильмы и так далее. Таким образом, управление контентом -- это процесс управления подобными материалами. Он включает следующие элементы: размещение материалов на сервере, удаление материалов с сервера, когда в них больше нет необходимости, организацию (реорганизацию) материалов, возможность отслеживать их состояние.

Системы управления контентом (Content Management Systems, CMS) – это программные комплексы, автоматизирующие процедуру управления контентом.

\emph{Цель настоящей работы} – разработка системы управления содержимым веб-сайтов. Для достижения поставленной цели необходимо решить \emph{следующие задачи:}
\begin{itemize}
\item провести анализ предметной области;
\item разработать концептуальную модель программно-информационной системы;
\item спроектировать и реализовать серверную и клиентскую часть программной системы средствами web-технологий;
\item провести тестирование работы программно-информационной системы.
\end{itemize}

\emph{Структура и объем работы.} Отчет состоит из введения, 4 разделов основной части, заключения, списка использованных источников, 2 приложений. Текст выпускной квалификационной работы равен \formbytotal{lastpage}{страниц}{е}{ам}{ам}.

\emph{Во введении} сформулирована цель работы, поставлены задачи разработки, описана структура работы, приведено краткое содержание каждого из разделов.

\emph{В первом разделе} приводится анализ предметной области.

\emph{Во втором разделе} на стадии технического задания приводятся требования к разрабатываемой программно-информационной системе.

\emph{В третьем разделе} на стадии технического проектирования представлены проектные решения для программно-информационной системы.

\emph{В четвертом разделе} приводится список классов и их методов, использованных при разработке серверной части программно-информационной системы, производится тестирование разработанного сайта.

В заключении излагаются основные результаты работы, полученные в ходе разработки.

В приложении А представлен графический материал.
В приложении Б представлены фрагменты исходного кода. 
