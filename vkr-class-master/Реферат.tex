\abstract{РЕФЕРАТ}

Объем работы равен \formbytotal{lastpage}{страниц}{е}{ам}{ам}. Работа содержит \formbytotal{figurecnt}{иллюстраци}{ю}{и}{й}, \formbytotal{tablecnt}{таблиц}{у}{ы}{}, \arabic{bibcount} библиографических источников и \formbytotal{числоПлакатов}{лист}{}{а}{ов} графического материала. Количество приложений – 2. Графический материал представлен в приложении А. Фрагменты исходного кода представлены в приложении Б.

Перечень ключевых слов: веб-сайт, веб-страница, cистема, CMS, веб-форма, Apache, веб-сервер, база данных, класс, компонент, модуль, сущность, модель, контроллер, представление, метод, редактор контента, администратор, пользователь.

Объектом разработки является программно-информационная система для управления содержимым веб-сайта.

Целью выпускной квалификационной работы является разработка системы управления содержимым веб-сайта, предназначенной для совместного создания и управления веб-сайтами.

В процессе создания системы были выделены основные сущности системы, разработана база данных. Была использована методология ООП, спроектированы классы и реализованы методы модулей, обеспечивающие работу с сущностями предметной области и корректную работу системы.

При разработке системы был использован веб-сервер Apache и языки программирования PHP, Javascript, язык разметы HTML, каскаданые таблицы стилей CSS.

\selectlanguage{english}
\abstract{ABSTRACT}
  
The volume of work is \formbytotal{lastpage}{page}{}{s}{s}. The work contains \formbytotal{figurecnt}{illustration}{}{s}{s}, \formbytotal{tablecnt}{table}{}{s}{s}, \arabic{bibcount} bibliographic sources and \formbytotal{числоПлакатов}{sheet}{}{s}{s} of graphic material. The number of applications is 2. The graphic material is presented in annex A. Source code fragments are presented in Appendix B.

List of keywords: website, web page, system, CMS, web form, Apache, web server, database , class, component, module, entity, model, controller, view, method, content editor, administrator, user.

The object of development is a software information system for managing website content.

The purpose of the final qualifying work is to develop a website content management system intended for joint creation and management of websites.

In the process of creating the system, the main entities of the system were identified and a database was developed. The OOP methodology was used, classes were designed and module methods were implemented to ensure work with domain entities and correct operation of the system.

When developing the system, the Apache web server and the programming languages ​​PHP, Javascript, HTML markup language, and CSS cascading style sheets were used.

\selectlanguage{russian}
